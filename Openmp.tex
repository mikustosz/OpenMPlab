\documentclass[11pt]{article}
\usepackage[english]{babel}
\usepackage{array} % for better arrays (eg matrices) in maths
\usepackage{booktabs} %Tablas quedan mejor
\usepackage{vmargin} %M�rgenes
\usepackage{hyperref}
\setpapersize{A4}
\setmargins{2.5cm}       % margen izquierdo
{1.5cm}                        % margen superior
{16.5cm}                      % anchura del texto
{22.42cm}                    % altura del texto
{10pt}                           % altura de los encabezados
{1.5cm}                           % espacio entre el texto y los encabezados
{0pt}                             % altura del pie de p�gina
{2cm}                           % espacio entre el texto y el pie de p�gina

\usepackage[none]{hyphenat} %No corta palabras al final de las lineas
\usepackage{graphicx} %Im�genes
\usepackage{subfigure} % subfiguras
\graphicspath{{img/}} % Root for images
\usepackage{wrapfig}
\usepackage{siunitx} %Escribir n�meros y unidades
\usepackage{amsmath} %Mayor facilidad para escribir matrices y otros elementos matem�ticos
%\numberwithin{equation}{section} %En las ecuaciones se especifica tambi�n el cap�tulo
\usepackage{xcolor} %Colores (\color{})
\usepackage{lipsum}
\usepackage{topcapt}
\usepackage{parskip}
\usepackage{braket}
\usepackage{setspace}
\usepackage{listings}
\usepackage{fancyhdr} %Cabeceras
\pagestyle{fancy} % seleccionamos un estilo


\definecolor{mygreen}{rgb}{0,0.6,0}
\definecolor{mygray}{rgb}{0.5,0.5,0.5}
\definecolor{mymauve}{rgb}{0.58,0,0.82}


\lstset{ %
  backgroundcolor=\color{white},   % choose the background color; you must add \usepackage{color} or \usepackage{xcolor}; should come as last argument
  basicstyle=\scriptsize,        % the size of the fonts that are used for the code
  breakatwhitespace=false,         % sets if automatic breaks should only happen at whitespace
  breaklines=true,                 % sets automatic line breaking
  captionpos=b,                    % sets the caption-position to bottom
  commentstyle=\color{mygreen},    % comment style
  deletekeywords={...},            % if you want to delete keywords from the given language
  escapeinside={\%*}{*)},          % if you want to add LaTeX within your code
  extendedchars=true,              % lets you use non-ASCII characters; for 8-bits encodings only, does not work with UTF-8
  frame=L,	                   % adds a frame around the code
  keepspaces=true,                 % keeps spaces in text, useful for keeping indentation of code (possibly needs columns=flexible)
  keywordstyle=\color{blue},       % keyword style
  language=C,                 % the language of the code
  morekeywords={*,...},           % if you want to add more keywords to the set
  numbers=left,                    % where to put the line-numbers; possible values are (none, left, right)
  numbersep=8pt,                   % how far the line-numbers are from the code
  numberstyle=\tiny\color{mygray}, % the style that is used for the line-numbers
  rulecolor=\color{black},         % if not set, the frame-color may be changed on line-breaks within not-black text (e.g. comments (green here))
  showspaces=false,                % show spaces everywhere adding particular underscores; it overrides 'showstringspaces'
  showstringspaces=false,          % underline spaces within strings only
  showtabs=false,                  % show tabs within strings adding particular underscores
  stepnumber=2,                    % the step between two line-numbers. If it's 1, each line will be numbered
  stringstyle=\color{mymauve},     % string literal style
  tabsize=2,	                   % sets default tabsize to 2 spaces
  title=\lstname                   % show the filename of files included with \lstinputlisting; also try caption instead of title
}


\lhead{Openmp programming} %CAMBIA
\rhead{Parallel programming} %CAMBIA





\begin{document}

\setstretch{1.3} % Line spacing of 1.5
\setlength{\parskip}{5mm}

\renewcommand{\tablename}{Table}

\thispagestyle{empty} %Primera p�gina sin n�mero
\begin{center}
	
	{\scshape\LARGE Parallel programming\par}
	\vspace{0.5cm}
	\rule{15cm}{0.8pt}\\
	{\huge\bfseries Openmp programming: Laplace\par}
	\rule{15cm}{0.8pt}\\
	\vspace{0.5cm}
	{\large \textbf{ppM-1-4:} \itshape Michal Kustosz, Jorge Pardillos\par}

\end{center}
\setcounter{page}{1} %Empezamos a contar los n�meros



\pagenumbering{arabic} %Ponemos n�meros normales (�rabes) ya
\pagestyle{fancy} % seleccionamos un estilo


\section{Introduction}

The aim of this report is to study the effect of parallelisation in \emph{C} using \emph{openmp}.

First of all, the code will be analysed and implementations will be made in order to parallelise it, commenting the initial results of such parallelisation.

After that, using the tool that has been installed during the lab sesions \emph{Tau}, a performance analysis will be carried out, studying more in detail the results of this parallelisation and its metrics.

\section{About the code}

The code to be analysed is a simulation of the Laplace equation. In this program, the main function resides in the function called \emph{laplace\_step()}, where the new elements of the matrix $A$ are computed based on the previous values.

More specifically, the following code is the core of the program:
\lstset{language=C}
\begin{lstlisting}		
for ( j=1; j < n-1; j++ )
    #pragma omp simd reduction(max:error)
    for ( i=1; i < n-1; i++ )
    {
      out[j*n+i]= stencil(in[j*n+i+1], in[j*n+i-1], in[(j-1)*n+i], in[(j+1)*n+i]);
      error = max_error( error, out[j*n+i], in[j*n+i] );
    }
\end{lstlisting}

In order to parallelise it, one pragma instruction has been introduced before the while loop. The variables $A$, $temp$ (containing the information about the matrices), \emph{error}, \emph{iter} and $n$ have been selected as shared variables. The number of threads is defined here as well. Inside the loop, the \emph{laplace\_step()} is executed in parallel in all the threads, while the rest of the code is executed sequentially.

\lstset{language=C}
\begin{lstlisting}		
#pragma omp parallel num_threads(8) shared(A,temp,error,iter,n)
  while ( error > tol*tol && iter < iter_max )
  {
    error= laplace_step (A, temp, n);
    #pragma omp master
    {
    iter++;
    float *swap= A; A=temp; temp= swap; // swap pointers A & temp
    }
  }
\end{lstlisting}

Inside the function \emph{laplace\_step()} there has been introduced the following code:
\begin{lstlisting}		
float laplace_step(float *in, float *out, int n)
{
  int i, j;
  float error=0.0f;
  #pragma omp for private(i, j) nowait
  for ( j=1; j < n-1; j++ )
    #pragma omp simd reduction(max:error)
    for ( i=1; i < n-1; i++ )
    {
      out[j*n+i]= stencil(in[j*n+i+1], in[j*n+i-1], in[(j-1)*n+i], in[(j+1)*n+i]);
      error = max_error( error, out[j*n+i], in[j*n+i] );
    }
  return error;
}
\end{lstlisting}

This way the program is properly parallelised, and we can proceed to measure its metrics with the tool \emph{perf}.

In the table below it is described the results obtained for this program. Improvements have been obtained with all the number of threads.

\newpage

\begin{table}[htbp]
\centering
\begin{tabular}{l c c c c c c c c r}
\toprule
Threads	&	Cycles	&	Instructions	&  Insns/cycle & task-clock	&	cache-misses & Seconds	\\
\midrule
Sequential & 82.358.161.316 & 117.854.234.964  & 1,43 & 24406,62 & 63.086.945 & 24,459\\
2 & 125.763.778.766 & 118.017.309.730 & 0,94 & 38838,00 & 304.712.719 & 19,472\\
4 & 224.244.846.876 & 118.029.183.954 & 0,53 & 69178,65 & 672.032.373 & 17,355\\
8 & 455.615.949.828 & 118.796.320.205 & 0,26 & 140745,87 & 1.158.142.860 & 17,681\\
\bottomrule							
\label{table:parallelcode}
\end{tabular}
\caption{Metric results using N=10000, iter=100. Compiler: gcc 6.1.0. $3.30$GHz}
\end{table}


However, it is important to remark that this results have been obtained using the optimisation provided by the compiler, i.e. using the \emph{-Ofast} command. This option introduces as many optimisations in de code as possible, leaving very small chance to improvements. If this option had not been used, the sequential program would have performed much worse, and the parallelisations would have been several times faster than the original one. Usually, as many times faster as the number of open threads.


\section{Tau analysis}

\subsection{Openmp}

\subsection{Openmp+papi}

\end{document}

























